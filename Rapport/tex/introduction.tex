\chapter{Introduction}
\label{sec:introduction}

% The Introduction is crucially important. By the time a referee has finished the Introduction, he's probably made an initial decision about whether to accept or reject the paper -- he'll read the rest of the paper looking for evidence to support his decision. A casual reader will continue on if the Introduction captivated him, and will set the paper aside otherwise. Again, the Introduction is crucially important.

% Here is the Stanford InfoLab's patented five-point structure for Introductions. Unless there's a good argument against it, the Introduction should consist of five paragraphs answering the following five questions:

% What is the problem?
The problem in this project is to analyse and develop an C\# .NET  application to manage UML class diagrams in a way so that it can be extended with other types of UML diagrams in the future.

% Why is it interesting and important?
This is interesting because different design patterns and methods can be used

% Why is it hard? (E.g., why do naive approaches fail?)



% Why hasn't it been solved before? (Or, what's wrong with previous proposed solutions? How does mine differ?)

% What are the key components of my approach and results? Also include any specific limitations.


% Then have a final paragraph or subsection: "Summary of Contributions". It should list the major contributions in bullet form, mentioning in which sections they can be found. This material doubles as an outline of the rest of the paper, saving space and eliminating redundancy.