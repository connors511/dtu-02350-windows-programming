\chapter{Introduction}
\label{sec:introduction}

% The Introduction is crucially important. By the time a referee has finished the Introduction, he's probably made an initial decision about whether to accept or reject the paper -- he'll read the rest of the paper looking for evidence to support his decision. A casual reader will continue on if the Introduction captivated him, and will set the paper aside otherwise. Again, the Introduction is crucially important.

% Here is the Stanford InfoLab's patented five-point structure for Introductions. Unless there's a good argument against it, the Introduction should consist of five paragraphs answering the following five questions:

% What is the problem?
The problem with this project is analysing the different kinds of data and visualize it in a smooth and easy way. Other weather applications exists on the market, but one of the preferences to create the new weather application, is using open source (OpenStreetMap, Qt etc.). 

The test data provided to this project is in different kinds of file formats, namely csv, wrk and NetCDF\footnote{Network Common Data Form - See \cite{netcdf}}.

% Why is it interesting and important?
This is interesting because there is a lot of data to analyze and visualize. A combination of different programming languages will be used to get higher performance while maintaining functionality and user-friendliness.

% Why is it hard? (E.g., why do naive approaches fail?)
Because of the many different file formats and huge amount of data, it is difficult to create an application that is flexible enough to handle all the data in an optimal way.

% Why hasn't it been solved before? (Or, what's wrong with previous proposed solutions? How does mine differ?)
There are other applications out there, like DMI\footnote{Danish Meteorological Institute - \url{http://dmi.dk}} and TV2-vejret\footnote{Denmark's nationwide commercial TV channel - \url{http://vejret.tv2.dk}}, but none of these satisfied our requirements that was specified at the beginning. DMI had many different data, but wasn't created in a user-friendly way and TV2-vejret only have weather forecast. There are similar weather programs out there, but none of these have either user-friendly way of handling huge amount data or using open source.

% What are the key components of my approach and results? Also include any specific limitations.
This project is going to focus on extensibility, speed and user experience, as key components.

% Then have a final paragraph or subsection: "Summary of Contributions". It should list the major contributions in bullet form, mentioning in which sections they can be found. This material doubles as an outline of the rest of the paper, saving space and eliminating redundancy.