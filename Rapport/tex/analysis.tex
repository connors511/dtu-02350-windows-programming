\chapter{Analysis}
\label{sec:analysis}
A lot of considerations have gone into the development process of the application. The following chapter explains the considerations with the most impact on the application.

TÆLLER 5

\section{Choice of diagram}
\label{sec:the_data}




\subsubsection{Solution strategy}
\label{sec:analysis_timezones}

iterationsplan

\section{Risks}
\label{sec:graphical_user_interface}

Fremgangsmåden ved risikovurderinger er:

    Identifikation af risici
    Vurdering af risici
    Handlinger mod risici
    Opfølgning


\section{Comparison}
\label{sec:comparison}
It has been discussed if it should be possible to compare data from multiple wind farms at the same time, and how it should be done if that was the case.

One of the ideas was to let single click on a windmill put it in a `selected' state and a double click on a windmill would trigger the comparison view between all the windmills in the `selected' state.

The comparison view was also discussed to be either one chart, like the one implemented, but with all the data represented with different colors for each wind farm.
Another way of viewing them was to have multiple, small, draggable windows, each one representing a wind farm. This would allow browsing the data for each wind farm independent of each other, but also make it annoying when wanting to compare at the same time and having to manually move back or forth in time on each window.

\section{Time handling}
\label{sec:time_handling}
As time goes by, one would expect the application to contain massive amounts of data.
In the provided test data\footnote{Data02122.tar.gz provided by Pierre-Julien Trombe on CampusNet}, the application contains \textasciitilde 140 rows of data for \textsf{wrk} files and \textasciitilde 73,000 rows of data for the \textsf{csv} file.

It was discussed how we should limit the view of data, since it would require quite a lot of processing power to display that much data.

A `from' and `to' field for selecting a date range that all data on the application would be limited to which would increase performance, reduce server load and increase overall UX.