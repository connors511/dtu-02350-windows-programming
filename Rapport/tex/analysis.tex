\chapter{Analysis}
\label{sec:analysis}
A lot of considerations have gone into the development process of the application. The following chapter explains the considerations with the most impact on the application.

TÆLLER 5

\section{Choice of diagram}
\label{sec:choice_diagram}

While state and sequence diagrams show how (part of) an application works in a detailed way, class diagrams give an overview over all or some classes in the application, what properties and methods they contain and how they are related to each other. In this way, class diagrams can be used to easily see how an application is modelled and thus how it can be altered or extended. Therefore, class diagrams are used more often than state and sequence diagrams when it comes to software architecture, and therefore an application to manage class diagrams has been chosen.

\subsection{Solution strategy}
\label{sec:sol_strat}

Due to limited time, not all specifications for a UML class diagram can be implemented in the application, and thus we have made a prioritized specification list, an iteration plan, where the first element is of highest priority and will be designed and implemented first.

\begin{enumerate}
  \item Different types of classes
  \item Undo and redo functionality
  \item CRUD for classes
  \item Movable classes
  \item Relations between classes
  \item Save and load functionality
  \item Automatic save functionality
\end{enumerate}

One should think that \textit{CRUD for classes} and \textit{Movable classes} is more important than \textit{Undo and redo functionality}, but as they are highly dependent on \textit{Undo and redo functionality}, this has higher priority. E.g. if \textit{CRUD for classes} was made before \textit{Undo and redo functionality}, one could risk having to modify it all again in order to get undo and redo to work properly in the end.

\subsection{Risks}

Because of the solution strategy where a functionality has been given a priority corresponding to its importance, the risks of not implementing one of the specifications due to limited time have been minimized. If \textit{Automatic save functionality} was left out, it would not cause the application to work improperly as it has no impact on the actual purpose - that is, creating an application that can manage UML class diagrams.\\
This is an advantage of using an iteration plan in the process of software development.
