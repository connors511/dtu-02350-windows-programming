\chapter{Design \& Implementation}
\label{sec:design_and_implementation}

\section{GUI}
\label{sec:gui}
TÆLLER 35

The GUI consists of a menu, a two-column grid and a status bar. The features that can be interacted with in the GUI have been outlined below.

\subsection{Editable classes}

There are different gestures for creating, updating and deleting classes, each chosen to get the best user experience. In the left grid column, one can click on a button to create a specific type of class in the application. At the moment there are four types to choose among:

\begin{itemize}
  \item Class
  \item Abstract Class
  \item Enum
  \item Struct
\end{itemize}

Clicking one of the buttons will also insert a box in the right grid column representing the class in the application. The color of the boxes are determined by the type of the classes.\\

Deleting a class is done by right clicking on the specific box representing that class.\\

If one wants to edit a class, this is done by double (left) clicking on the specific box. This will cause the box to switch to an editable field where one can type in the name, properties and methods for the class in accordance with the syntax of UML class diagrams. Clicking on the same box again or double clicking on another box will save the changes, and the boxes will automatically resize to fit its content. Thus, only one class can be edited at a time.\\
Choosing inline text edit instead of a pop-up window or likewise is because the UML class diagram content syntax is so simple and understandable and because it (for a software architect's point of view) is much faster to use this text based UI than pick the different properties and methods etc. in a GUI.

\subsubsection{Regex}

\subsection{Movable classes}

It is possible to freely move the boxes around in the right grid column, but only there. That means that it is not possible to e.g. move the boxes out of the window or into the left grid column.\\
When editing a box, one cannot move it, but it is still possible to move other boxes.

\subsection{Undo and redo}

\subsection{Relations}

Relations will be created, updated and deleted automatically based on the classes (boxes) present.

\subsection{Menu and shortcuts}

In the top menu, one can find various options and functions. Each of them has been given a shortcut key combination that correspond to the default shortcut for its type (if it exists). The options and their shortcuts can be seen in table \ref{tab:menu_shortcut}.

\begin{table}[htbp]
\centering
\begin{tabular}{|l|l|}
\hline
\textbf{Option} & \textbf{Shortcut}\\
\hline
New file & Ctrl+N\\
\hline
Open file & Ctrl+O\\
\hline
Save file & Ctrl+S\\
\hline
Save as & Ctrl+A\\
\hline
Undo & Ctrl+Z\\
\hline
Redo & Ctrl+Y\\
\hline
\end{tabular}
\caption{Shortcuts of menu options}
\label{tab:menu_shortcut}
\end{table}

\subsection{Status bar}

On the bottom of the GUI a status bar can be seen. Here, the user gets messages about what the status is, i.e. what is going on in the application. E.g. if a box is being edited, removed, moved or if a file is being (automatically) saved or loaded.

\section{Save and load}
\label{sec:design_pattern}

\section{Application layer}
\label{sec:app_layer}

TÆLLER 20

dll
different class types
thread

\section{Design patterns}
\label{sec:design_pattern}

2-3

TÆLLER 10


\section{Tests}
\label{sec:tests}

TÆLLER 10

\section{Noget der ikke er udnervist i}


TÆLLER 5